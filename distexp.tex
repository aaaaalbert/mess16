%!TEX root = mess16.tex
\section{Running Distributed Experiments}\label{sec:distexp}

In addition to providing a flexible programmable sensor laboratory 
for local experimentation, \sys is designed to allow for including 
contributed (``donated'') resources in distributed experiments as well. 
For this to work, a Clearinghouse is introduced as a trusted 
intermediary between volunteers that contribute computational and 
sensor resources on their devices, and researches wanting to run 
experiments on multiple, distributed sandboxes.

Clearly, an experiment must not be able to perform privacy-invasive 
tasks such as reading the exact GPS location of the device, unless 
the device owner consents. We chose to implement a fine-grained 
method for codifying consent in the form of sensor access restriction 
policies: 
Both the accuracy of the sensor value and the frequency of readout 
can be limited. This gives device owners total control over the 
amount of sensor data they are comfortable to share with an 
experimenter.

However, the distributed mode of experimentation does not only rely on 
device owners to protect their privacy. We also demand from researchers 
that their institutional review boards (IRBs) examine and approve the 
experiment design. From this, an additional stack of policies is generated, 
which provides privacy protection in case no other, more stringent 
owner-set policies exist.
More details on the implementation of policies can be found in our 
prior contribution \cite{zhuang2015privacy}.
