%!TEX root = mess16.tex
\section{Overview}\label{sec:design}

\sys is an app for Android devices such as smartphones and tablets. 
Once configured and installed, it presents to the experimenter a 
programmable, sensor-enabled Python-based sandbox environment that 
is remote-controllable using TCP/IP networking.


\subsection{The \sys sandbox}

The sandbox is \sys's execution environment for experimenter code. 
Inside the sandbox, experimenters can use a restricted, performance-isolated 
subset of the \textsc{Python 2} high-level scripting language to 
read the device's sensors, perform computations, 
store and retrieve data files, use TCP and UDP sockets, and so 
on\footnote{Sandbox API documentation: \url{https://seattle.poly.edu/wiki/RepyV2API}}$^,$\footnote{Sensor API documentation: \url{https://sensibilitytestbed.com/projects/project/wiki/sensors}}. 
An extensive set of libraries\footnote{Library sources: \url{https://github.com/SensibilityTestbed/seattlelib_v2}} extends the sandbox's 
capabilities, and 
provides additional mathematical functions, matrix manipulation, 
network time synchronization via NTP, crypto, an HTTP server, etc.

The restrictions placed on the sandbox make it so that an experiment 
is confined to its specific directory (and cannot traverse the filesystem); 
furthermore, only a small number of non-privileged ports is allowed for 
network traffic. In terms of performance isolation, an experiment's 
allotment of CPU share is capped, as are the file read and write rates 
and the available network bandwidth.
Isolation and restrictions are of particular importance when the 
experimenter runs code on devices he or she does not own; see 
Section~\ref{sec:distexp} for an introduction to these use cases.



Table~\ref{tbl:sensors} overviews the sensors that \sys currently 
exposes on a typical device. We take ``sensor'' in this context to mean 
any data source that provides information about the physical environment 
that the device is in, including for example wireless networks. 
Newer devices expose larger sets of sensors (e.g. including a 
barometer, a step counter, and NFC), which we strive to add to the 
sandbox as their prevalence increases. 
As noted previously, \sys's sandbox simplifies Android's approach of 
handling sensors and provides function calls to let an experiment 
query sensor values directly.

\begin{table}
\small
\caption{Sensors that \sys currently exposes}\label{tbl:sensors}
\centering
\begin{tabular}{c|c}
\hline
Sensor & Quantity sensed [unit (where applicable)]\\
\hline
\hline
Accelerometer & acceleration [\si{\meter\per\square\second}], 3 axes\\
Magnetometer & magnetic field [\si{\micro\tesla}], 3 axes\\
Gyroscope & rotation [\si{\radian\per\second}], 3 axes\\
Ambient light & illuminance [\si{\lux}]\\
\hline
Battery & Voltage [\si{\volt}], temperature [\si{\degreeCelsius}], \ldots\\
WLAN & (B)SSID, RSSI [\si{dB}m], MAC address, \ldots\\
Cellular network & Cell ID, LAC, network type, \ldots\\
%Bluetooth
%NFC
GPS & latitude [\si{\degree}], longitude [\si{\degree}], altitude [\si{\meter}], \ldots\\
\hline
QR code reader & arbitrary data\\
\hline
\end{tabular}
\end{table}


\subsection{Interfacing with the sandbox}
The experimenter uses an experiment manager tool installed on his 
or her computer to control the sandbox on the Android device and to 
upload or download code and data.
The experiment management tool communicates with the app using a 
cryptographically signed protocol over TCP/IP networking (e.g. WLAN), 
thereby obviating the need for dedicated interfaces like I2C, RS323, 
or JTAG.
% Internet connectivity, NAT traversal

The \sys app itself is designed to run continuously in the background, 
not providing an interactive graphical user interface beyond what is 
required for configuration purposes. This enables unobtrusive distributed 
experimentation on ``donated'' resources (i.e. installs on devices 
not owned by actual experimenters), as 
Section~\ref{sec:distexp} below discusses. 
However, there are multiple ways of providing input to a sandbox. 
One source of input is the experiment management tool. 
Then, the previously mentioned sensors may provide data. For example, 
scanning a QR code that uses \sys's custom URI scheme 
``\texttt{sensibilitytestbed://}'' results in the URI's path to 
be provided to the sandbox.
% For example, scanning a QR code that decodes to \texttt{sensibilitytestbed://sensibilitytestbed.com/YOUR\_DATA} provides \texttt{YOUR\_DATA} to the sandbox.
One can also transfer data over plain 
TCP or UDP sockets, or even implement ``web apps'' by 
running a small local web server in the sandbox. 
When opened in the device's browser, the content served can be used to 
read user input, provide feedback, visualize sensor readings, and so on.

Furthermore, certain basic output methods can be enabled for a sandbox. 
These include actuating the devices's vibrator, using the text-to-speech 
interface, and presenting an on-screen notification popup.
% (or \texttt{Toast} in Android's terminology\footnote{\url{https://developer.android.com/guide/topics/ui/notifiers/toasts.html}}). 
This is particularly useful for 
experiments that should deliver immediate feedback, or when using a separate 
computer for control is unwieldy.


\subsection{An illustrative example}

Suppose that we want to measure the magnitude of the acceleration 
vector acting on a device running \sys.
Listing~\ref{alg:log-accel} presents actual, working example code.
It polls the device's accelerometers, receiving three values that correspond 
to the spatial axes, calculates the vector's magnitude, and logs the 
timestamped values. Note that ``\texttt{**}'' is Python's exponentiation 
operator, used here for squaring and taking the square root; 
\texttt{get\_acceleration}, \texttt{log} and \texttt{getruntime} are 
API calls defined by the \sys sandbox.

Since the \sys app takes care of initializing the sensors and encapsulates 
Android's native sensor facilities, experiments are simple to bootstrap, 
and the entry threshold is low. Also, updating an experiment is only a 
matter of uploading new code to the sandbox and restarting it: The \sys 
app remains unchanged and need not be reinstalled, and no build or 
packaging steps are required to push out updated code. This eases 
development and debugging, making \sys a great platform for prototyping 
and iterative development.

\begin{lstlisting}[language=Python,float,numbers=left,basicstyle=\small,emph={get_acceleration},xleftmargin=2em,emphstyle=\textit,label=alg:log-accel,caption=\textsc{Code sample: Log the magnitude of acceleration}]
STANDARD_GRAVITY = 9.81
while True:
  acc_x, acc_y, acc_z = get_acceleration()
  magnitude = ((acc_x-STANDARD_GRAVITY)**2 + 
      acc_y**2 + acc_z**2) ** 0.5
  log(getruntime(), magnitude, "\n")
\end{lstlisting}
