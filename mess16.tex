
%% This file is based on bare_conf.tex
%% V1.4b
%% 2015/08/26
%% by Michael Shell
%% See:
%% http://www.michaelshell.org/
%% for current contact information.
%%
%% This is a skeleton file demonstrating the use of IEEEtran.cls
%% (requires IEEEtran.cls version 1.8b or later) with an IEEE
%% conference paper.
%%
%% Support sites:
%% http://www.michaelshell.org/tex/ieeetran/
%% http://www.ctan.org/pkg/ieeetran
%% and
%% http://www.ieee.org/

%%*************************************************************************
%% Legal Notice:
%% This code is offered as-is without any warranty either expressed or
%% implied; without even the implied warranty of MERCHANTABILITY or
%% FITNESS FOR A PARTICULAR PURPOSE! 
%% User assumes all risk.
%% In no event shall the IEEE or any contributor to this code be liable for
%% any damages or losses, including, but not limited to, incidental,
%% consequential, or any other damages, resulting from the use or misuse
%% of any information contained here.
%%
%% All comments are the opinions of their respective authors and are not
%% necessarily endorsed by the IEEE.
%%
%% This work is distributed under the LaTeX Project Public License (LPPL)
%% ( http://www.latex-project.org/ ) version 1.3, and may be freely used,
%% distributed and modified. A copy of the LPPL, version 1.3, is included
%% in the base LaTeX documentation of all distributions of LaTeX released
%% 2003/12/01 or later.
%% Retain all contribution notices and credits.
%% ** Modified files should be clearly indicated as such, including  **
%% ** renaming them and changing author support contact information. **
%%*************************************************************************


\documentclass[conference]{IEEEtran}

\usepackage{graphicx}

\usepackage{algorithmic}

\usepackage{url}

\usepackage{hyperref}

\usepackage{listings}

\usepackage{longtable}

\usepackage{tabu}

\usepackage[load-configurations={abbreviations,binary}]{siunitx}

\usepackage{xspace}

\newcommand{\sys}{Sensibility Testbed\xspace}

% correct bad hyphenation here
\hyphenation{op-tical net-works semi-conduc-tor}


\begin{document}
%
% paper title
% Titles are generally capitalized except for words such as a, an, and, as,
% at, but, by, for, in, nor, of, on, or, the, to and up, which are usually
% not capitalized unless they are the first or last word of the title.
% Linebreaks \\ can be used within to get better formatting as desired.
% Do not put math or special symbols in the title.
\title{\sys: A Programmable Sensor Laboratory For Android Devices}


% author names and affiliations
% use a multiple column layout for up to three different
% affiliations
\author{\IEEEauthorblockN{Albert Rafetseder, Yanyan Zhuang, and Justin Cappos}
\IEEEauthorblockA{NYU Tandon School of Engineering\\
6 MetroTech Center, Brooklyn, NY 11201%\\
%Email.
}
}

% make the title area
\maketitle

%!TEX root = mess16.tex
\begin{abstract}
We present \sys\footnote{\url{https://sensibilitytestbed.com}}, 
a programmable sensor laboratory that runs on Android 
devices such as smartphones and tablets. \sys leverages the computational 
power and sensor facilities on these prevalent devices, and provides an 
easy-to-use, high-level programmable sandbox to enable experimentation 
and prototyping of sensor-enabled code.
\sys is an active project used by students, educators and researchers 
all around the world to experiment with sensor-enabled systems and 
advance the state of the art in sensor algorithms.
% All components of \sys are free and open-source software (FOSS). 
\end{abstract}


% For peer review papers, you can put extra information on the cover
% page as needed:
% \ifCLASSOPTIONpeerreview
% \begin{center} \bfseries EDICS Category: 3-BBND \end{center}
% \fi
%
% For peerreview papers, this IEEEtran command inserts a page break and
% creates the second title. It will be ignored for other modes.
\IEEEpeerreviewmaketitle

%!TEX root = mess16.tex
\section{Introduction}

Today's smartphones and tablets are sensor-rich, computationally 
powerful, highly integrated devices with huge amounts of storage 
capacity, can run on battery for extended periods of time, and are often 
connected to the Internet via wireless networks.
Also, commoditization has driven down the price of devices, making them 
readily available.

We thus view smartphones and tablets as an attractive platform to 
prototype sensor-enabled experiments on, and our \sys Android 
app\footnote{\url{https://play.google.com/store/apps/details?id=com.sensibility_testbed}} 
exists to support experimenters to do so. For this, \sys provides a 
high-level programmable sandbox environment that encapsulates and 
abstracts away many of the underlying operating system's peculiarities. 

Experiment code queries sensor values through function calls. 
This is a huge simplification over Android's native approach of registering 
listeners, receiving event notifications, and so on\footnote{For an overview of sensor usage in Android, please see the introductory material at \url{https://developer.android.com/guide/topics/sensors/sensors_overview.html}}.
Furthermore, the code development cycle is accelerated: Replacing the 
experiment code in the sandbox is a matter of uploading a new source 
file to the device. There are no build or packaging steps.

The rest of this exposition mainly focuses on \sys's design, 
explains how to interface with the sandbox, and discusses a small piece 
of working example code. We also touch on techniques we built into \sys 
that allow for privacy-preserving distributed experimentation.


%!TEX root = mess16.tex
\section{Overview}\label{sec:design}

\sys is an app for Android devices such as smartphones and tablets. 
Once configured and installed, it presents to the experimenter a 
programmable, sensor-enabled Python-based sandbox environment that 
is remote-controllable using TCP/IP networking.


\subsection{The \sys sandbox}

The sandbox is \sys's execution environment for experimenter code. 
Inside the sandbox, experimenters can use a restricted, performance-isolated 
subset of the \textsc{Python 2} high-level scripting language to 
read the device's sensors, perform computations, 
store and retrieve data files, use TCP and UDP sockets, and so 
on\footnote{Sandbox API documentation: \url{https://seattle.poly.edu/wiki/RepyV2API}}$^,$\footnote{Sensor API documentation: \url{https://sensibilitytestbed.com/projects/project/wiki/sensors}}. 
An extensive set of libraries\footnote{Library sources: \url{https://github.com/SensibilityTestbed/seattlelib_v2}} extends the sandbox's 
capabilities, and 
provides additional mathematical functions, matrix manipulation, 
network time synchronization via NTP, crypto, an HTTP server, etc.

The restrictions placed on the sandbox make it so that an experiment 
is confined to its specific directory (and cannot traverse the filesystem); 
furthermore, only a small number of non-privileged ports is allowed for 
network traffic. In terms of performance isolation, an experiment's 
allotment of CPU share is capped, as are the file read and write rates 
and the available network bandwidth.
Isolation and restrictions are of particular importance when the 
experimenter runs code on devices he or she does not own; see 
Section~\ref{sec:distexp} for an introduction to these use cases.



Table~\ref{tbl:sensors} overviews the sensors that \sys currently 
exposes on a typical device. We take ``sensor'' in this context to mean 
any data source that provides information about the physical environment 
that the device is in, including for example wireless networks. 
Newer devices expose larger sets of sensors (e.g. including a 
barometer, a step counter, and NFC), which we strive to add to the 
sandbox as their prevalence increases. 
As noted previously, \sys's sandbox simplifies Android's approach of 
handling sensors and provides function calls to let an experiment 
query sensor values directly.

\begin{table}
\small
\caption{Sensors that \sys currently exposes}\label{tbl:sensors}
\centering
\begin{tabular}{c|c}
\hline
Sensor & Quantity sensed [unit (where applicable)]\\
\hline
\hline
Accelerometer & acceleration [\si{\meter\per\square\second}], 3 axes\\
Magnetometer & magnetic field [\si{\micro\tesla}], 3 axes\\
Gyroscope & rotation [\si{\radian\per\second}], 3 axes\\
Ambient light & illuminance [\si{\lux}]\\
\hline
Battery & Voltage [\si{\volt}], temperature [\si{\degreeCelsius}], \ldots\\
WLAN & (B)SSID, RSSI [\si{dB}m], MAC address, \ldots\\
Cellular network & Cell ID, LAC, network type, \ldots\\
%Bluetooth
%NFC
GPS & latitude [\si{\degree}], longitude [\si{\degree}], altitude [\si{\meter}], \ldots\\
\hline
QR code reader & arbitrary data\\
\hline
\end{tabular}
\end{table}


\subsection{Interfacing with the sandbox}
The experimenter uses an experiment manager tool installed on his 
or her computer to control the sandbox on the Android device and to 
upload or download code and data.
The experiment management tool communicates with the app using a 
cryptographically signed protocol over TCP/IP networking (e.g. WLAN), 
thereby obviating the need for dedicated interfaces like I2C, RS323, 
or JTAG.
% Internet connectivity, NAT traversal

The \sys app itself is designed to run continuously in the background, 
not providing an interactive graphical user interface beyond what is 
required for configuration purposes. This enables unobtrusive distributed 
experimentation on ``donated'' resources (i.e. installs on devices 
not owned by actual experimenters), as 
Section~\ref{sec:distexp} below discusses. 
However, there are multiple ways of providing input to a sandbox. 
One source of input is the experiment management tool. 
Then, the previously mentioned sensors may provide data. For example, 
scanning a QR code that uses \sys's custom URI scheme 
``\texttt{sensibilitytestbed://}'' results in the URI's path to 
be provided to the sandbox.
% For example, scanning a QR code that decodes to \texttt{sensibilitytestbed://sensibilitytestbed.com/YOUR\_DATA} provides \texttt{YOUR\_DATA} to the sandbox.
One can also transfer data over plain 
TCP or UDP sockets, or even implement ``web apps'' by 
running a small local web server in the sandbox. 
When opened in the device's browser, the content served can be used to 
read user input, provide feedback, visualize sensor readings, and so on.

Furthermore, certain basic output methods can be enabled for a sandbox. 
These include actuating the devices's vibrator, using the text-to-speech 
interface, and presenting an on-screen notification popup.
% (or \texttt{Toast} in Android's terminology\footnote{\url{https://developer.android.com/guide/topics/ui/notifiers/toasts.html}}). 
This is particularly useful for 
experiments that should deliver immediate feedback, or when using a separate 
computer for control is unwieldy.


\subsection{An illustrative example}

Suppose that we want to measure the magnitude of the acceleration 
vector acting on a device running \sys.
Listing~\ref{alg:log-accel} presents actual, working example code.
It polls the device's accelerometers, receiving three values that correspond 
to the spatial axes, calculates the vector's magnitude, and logs the 
timestamped values. Note that ``\texttt{**}'' is Python's exponentiation 
operator, used here for squaring and taking the square root; 
\texttt{get\_acceleration}, \texttt{log} and \texttt{getruntime} are 
API calls defined by the \sys sandbox.

Since the \sys app takes care of initializing the sensors and encapsulates 
Android's native sensor facilities, experiments are simple to bootstrap, 
and the entry threshold is low. Also, updating an experiment is only a 
matter of uploading new code to the sandbox and restarting it: The \sys 
app remains unchanged and need not be reinstalled, and no build or 
packaging steps are required to push out updated code. This eases 
development and debugging, making \sys a great platform for prototyping 
and iterative development.

\begin{lstlisting}[language=Python,float,numbers=left,basicstyle=\small,emph={get_acceleration},xleftmargin=2em,emphstyle=\textit,label=alg:log-accel,caption=\textsc{Code sample: Log the magnitude of acceleration}]
STANDARD_GRAVITY = 9.81
while True:
  acc_x, acc_y, acc_z = get_acceleration()
  magnitude = ((acc_x-STANDARD_GRAVITY)**2 + 
      acc_y**2 + acc_z**2) ** 0.5
  log(getruntime(), magnitude, "\n")
\end{lstlisting}


%!TEX root = mess16.tex
\section{Running Distributed Experiments}\label{sec:distexp}

In addition to providing a flexible programmable sensor laboratory 
for local experimentation, \sys is designed to allow for including 
contributed (``donated'') resources in distributed experiments as well. 
For this to work, a Clearinghouse is introduced as a trusted 
intermediary between volunteers that contribute computational and 
sensor resources on their devices, and researches wanting to run 
experiments on multiple, distributed sandboxes.

Clearly, an experiment must not be able to perform privacy-invasive 
tasks such as reading the exact GPS location of the device, unless 
the device owner consents. We chose to implement a fine-grained 
method for codifying consent in the form of sensor access restriction 
policies: 
Both the accuracy of the sensor value and the frequency of readout 
can be limited. This gives device owners total control over the 
amount of sensor data they are comfortable to share with an 
experimenter.

However, the distributed mode of experimentation does not only rely on 
device owners to protect their privacy. We also demand from researchers 
that their institutional review boards (IRBs) examine and approve the 
experiment design. From this, an additional stack of policies is generated, 
which provides privacy protection in case no other, more stringent 
owner-set policies exist.
More details on the implementation of policies can be found in our 
prior contribution \cite{zhuang2015privacy}.


%!TEX root = mess16.tex
\section{Conclusion}\label{sec:conclusion}

This work presents \sys, 
an Android app implementing a programmable 
sensor laboratory. \sys build on today's powerful, sensor-rich 
smartphone and tablet devices. Its sandbox execution environment is 
based on the Python scripting language and provides an easy-to-use 
programming platform for both local and distributed experimentation.

\sys is actively used in education and academia.
Its components are free, open-source software (FOSS)\footnote{\url{https://github.com/SensibilityTestbed}}. 
We invite you to freely extend and reuse them.


%\begin{figure}[!t]
%\centering
%\includegraphics[width=2.5in]{myfigure}
%\caption{Simulation results for the network.}
%\label{fig_sim}
%\end{figure}


%\begin{table}[!t]
%% increase table row spacing, adjust to taste
%\renewcommand{\arraystretch}{1.3}
% if using array.sty, it might be a good idea to tweak the value of
% \extrarowheight as needed to properly center the text within the cells
%\caption{An Example of a Table}
%\label{table_example}
%\centering
%% Some packages, such as MDW tools, offer better commands for making tables
%% than the plain LaTeX2e tabular which is used here.
%\begin{tabular}{|c||c|}
%\hline
%One & Two\\
%\hline
%Three & Four\\
%\hline
%\end{tabular}
%\end{table}



% trigger a \newpage just before the given reference
% number - used to balance the columns on the last page
% adjust value as needed - may need to be readjusted if
% the document is modified later
%\IEEEtriggeratref{8}
% The "triggered" command can be changed if desired:
%\IEEEtriggercmd{\enlargethispage{-5in}}




\bibliographystyle{IEEEtran}
\bibliography{literature}




% that's all folks
\end{document}


