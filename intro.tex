%!TEX root = mess16.tex
\section{Introduction}

Today's smartphones and tablets are sensor-rich, computationally 
powerful, highly integrated devices with huge amounts of storage 
capacity, can run on battery for extended periods of time, and are often 
connected to the Internet via wireless networks.
Also, commoditization has driven down the price of devices, making them 
readily available.

We thus view smartphones and tablets as an attractive platform to 
prototype sensor-enabled experiments on, and our \sys Android 
app\footnote{\url{https://play.google.com/store/apps/details?id=com.sensibility_testbed}} 
exists to support experimenters to do so. For this, \sys provides a 
high-level programmable sandbox environment that encapsulates and 
abstracts away many of the underlying operating system's peculiarities. 

Experiment code queries sensor values through function calls. 
This is a huge simplification over Android's native approach of registering 
listeners, receiving event notifications, and so on\footnote{For an overview of sensor usage in Android, please see the introductory material at \url{https://developer.android.com/guide/topics/sensors/sensors_overview.html}}.
Furthermore, the code development cycle is accelerated: Replacing the 
experiment code in the sandbox is a matter of uploading a new source 
file to the device. There are no build or packaging steps.

The rest of this exposition mainly focuses on \sys's design, 
explains how to interface with the sandbox, and discusses a small piece 
of working example code. We also touch on techniques we built into \sys 
that allow for privacy-preserving distributed experimentation.
